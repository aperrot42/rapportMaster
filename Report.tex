\documentclass[]{report}   % list options between brackets


% Latex Police and file encoding
\usepackage[utf8]{inputenc}
\usepackage[T1]{fontenc}  % for character encoding 
\usepackage{lmodern}  % Police setting
\usepackage[american]{babel}


%style of tex :
\setlength{\parskip}{1ex plus 0.5ex minus 0.2ex} %space between paragraphs

%mathfonts
\usepackage{amsfonts}

% Code Formating
\input{settings/TexCodeSet}

%Inliine listings :
\usepackage{paralist}

%figures :
\usepackage{graphicx}
\usepackage{float}
% wrapped
\usepackage{wrapfig}
%\floatstyle{boxed} 
\restylefloat{figure}

%\usepackage[format=plain,indention=0cm,parskip=5pt,bf]{caption}
%subfigures 
 


%arays
\usepackage{array}

%
\def \gofigure{Gofigure\raisebox{-0.2ex}{2}}

%TODO macro
\newcommand{\TODO}[1] {\marginpar{\colorbox{red}{\Huge \textbf{TODO}}} \colorbox{red}{\Huge \textbf{TODO}}\colorbox{green}{\normalsize #1  } } 

\newcommand{\etal} {\textit{et al.}}

\usepackage{subfig}
\captionsetup[subfigure]{style=default, margin=5pt, parskip=0pt, hangindent=0pt, indention=0pt, singlelinecheck=true}
% Hypertext links
\usepackage[colorlinks=true,breaklinks=false,dvips,ps2pdf]{hyperref}
\urlstyle{sf}

%Maths
\usepackage{amsmath}




\begin{document}

\title{Master Thesis} 
\author{Antonin PERROT-AUDET
  Institut National des Sciences Appliquées,\\
  Lyon,\\
  France,\\
  \texttt{antonin.perrot-audet@insa-lyon.fr}}   
\date{February-September, 2010} 
\maketitle


\begin{abstract}
  This repport present the work carried out during my Master's Thesis in the Megason Lab, Harvard Medical School, Boston MA, USA.
  This internship is the last step of my formation in the National Institute of Applied Sciences of Lyon (INSA), and at the University Claude Bernard Lyon 1 (UCLB), to get the Master of Sciences in Electrical engineering and Proceeds, option Systems and Images, and the Master of Engineering in Electrical Engineering and Computer Sciences.

  The work presented here is based on microscopy images processing. I have been working on three dimension plus time datasets representing a developing zebra fish embryo. 
  The laboratory employs a team of computer scientists in charge of development of a program for biological data visualization and processing :{\gofigure}\cite{refGofigure2}.
  The project carried out while in the Megason Lab intend to be integrated to this program.
  
\tableofcontents  
  
 
\end{abstract}


\chapter{Description of the Megason Lab and its objectives} 
\phantomsection
\addcontentsline{toc}{chapter}{Description of the Megason Lab and its objectives} 

\section{The Megason Lab}

\subsection{The team}

The Megason Lab's employee are post-doc searchers. Two domains of expertise are present : Biology and informatics.
The biology team leads researches on the development of zebra fishes,
and the computer team works on developing a program for visualization, segmentation, and tracking of cells adapted to microscopy data.

{\small \begin{tabular*}{1.0\textwidth}{@{\extracolsep{\fill}} |  p{2.5cm} |  p{3.cm} | p{3.5cm} | c | }
\hline Nom & Poste & Intérêts & Nationalité \\ 
\hline Sean Megason & Professor & Financement et management du laboratoire. & USA \\ 
\hline Ramil Noche & Post-doctoral fellow & Formation du système nerveux. & Philipines \\ 
\hline Fengzhu Xiong & Graduate student & Création d'un poisson ayant une couleur différente par cellule. & Chine \\ 
\hline Nikolaus Obholzer & Post-doctoral fellow  & Développement de l'oreille et création de marqueurs génétiques fluorescents. &  Allemange \\ 
\hline Paul Cowgill & Graduate student & Effet des drogues sur le poisson zèbre. & USA \\ 
\hline David Tulga & Graduate student & Développement cellulaire. &  USA \\ 
\hline Ian Swiburne & Post-doctoral fellow & Développement de l'oreille du poisson zèbre. & USA \\ 
\hline Andrea Tentner & Post-doctoral fellow  & Différenciation des cellules provenant d'un ancêtre commun. & USA \\ 
\hline Amelia Green & Post-doctoral fellow  & Formation des oreilles du poisson zèbre. & Angleterre \\ 
\hline Evan Schwab & Associate & Séquences du code génétique du poisson zèbre. & USA \\ 
\hline Dante D'India & Technicien & Animal care. & USA \\ 
\hline Arnaud Gelas & Research Engineer (senior) &  Developing {\gofigure}: Project management. &  France \\
\hline Kishore Mosaliganti & Post-doctoral fellow  & Image processing algorithms, registration and segmentation. & India \\ 
\hline Lydie Souhait & Research Engineer & Developing {\gofigure}: Responsible de la base de donnees (design, connexion, requetes), et GUI & France \\
\hline Nicolas Rannou & Research Engineer &  Developing {\gofigure}: Responsible de la visualisation dans {\gofigure} & France \\
\hline Antonin Perrrot-Audet & Intern& Image processing : cells detection; developing {\gofigure} & France \\
\hline 
\end{tabular*} }

Note theat the Megason Lab is part of Systems biology, Harvard Medical School, and has access to its infrastructure and services.


\subsection{Objectives of the lab}

The main goal of the Megason lab is to understand the first steps of animal development.
The principle is to assimilate the genome with an informatics program.
By modifying the gene code of living organisms, and by observing the cellular development,
biologists can identify the actors of embryonic development, and understand their function.

Experiences are carried out on zebra fishes. The same way a computer scientist would introduce "breakpoints" in an informatic program to control its execution,
biologists create mutants and analyse the effects of genome modifications on the development.
The next step will consist in cerating a model of zebra fishes development.
Biologists are currently interrested in ear and nervous system formation.

Biologists acquire tremendous amount of data corresponding in three dimensional videos of zebra fishes embryos (typically, 150 Giga bytes per video).
In order to exploit this data, a team of developers has been formed. They are working on the next platform of microscopy omages analysis: {\gofigure}. 
The goal is to create a very accessible applmication ("cross-platform
\footnote{In computing, cross-platform, or multi-platform, is an attribute conferred to computer software or computing methods and concepts that are implemented and inter-operate on multiple operating systems and hardware.}
"), 
with a high quality code, freely distributed ("open-source
\footnote{of or relating to or being computer software for which the source code is freely available, but which use may be restricted by a license.}
"),
to provide other computer scientists with the opportunity to contribute and use the software.


\subsection{Hardware} 

The Megason Lab has state of the art equipments of microscopy and informatics. Harvard personnel is responsible for its servicing.
During the internship, I was especially interrested in the acquisition and image processing equipments :microscopes and computers.

\subsubsection{Imaging equipment}


The Megason lab has several microscopes for biologists to experiment on zebra fishes. The image processing team is especially interested in images produced by the two-photon, confocal and automated microscope.

Imageing technics in the Megason Lab are based on fluorescent markers fixed on certain molecules of the specimen to be imaged( the zebra fish embryo, at different development stages). Those phosphors' atoms must be excited in order to re-emit light.

The confocal microscope principle ( illustrated figure~\ref{fig:ConfocalPrinciple}) consists in illuminating (exciting) only the observed point: the illumination lens is also the observation lens. The imaged plan is the focal plan of this lens. Thus, the light energy is concentrated on the observed plan (focla plan). Moreover, in order to limitate the inpact of non observed plans, a diaphragm is added to the observation objective.

The confocal microscope of the Megason Lab is a two-photons microscope (figure~\ref{fig:Confocal2photonsPrinciple}).
The observed point is excited by two convergent laser rays.
This technic has several advantages :
\begin{itemize}
  \item The phosphor needs the energy from two photons to emit light. Outside of the focal point, there is never enough energy : the oly atoms able to emit light are the one of the focalized phosphors. This provides us with a better resolution.
  \item The excitation wavelength is approximatively twice the excitation wavelength of a traditional confocal microscope, and thus, can penetrate tissues better: we can analyses thicker specimens. 
  \item Each ray carries twice less energy, thus damaging less the specimen.
\end{itemize}

As of now, we can use three different markers at the time, that we store in a multichannel image.
This microscope has been assembled in the Megasaon Lab from a Zeiss 710 NLO (figure~\ref{fig:MicMegason}).

An environmental chamber has been created, in order to control embryo development while imaging.

\begin{figure}[htb]
\begin{center}
\leavevmode
\includegraphics[width=0.95\textwidth]{pictures/ConfocalPrinciple}
\end{center}
\caption{Confocal microscope principle (source :\href{http://en.wikipedia.org/wiki/File:Confocalprinciple.svg}{Wikipedia})}
\label{fig:ConfocalPrinciple}
\end{figure}

\begin{figure}[htb]
\begin{center}
\leavevmode
\includegraphics[width=0.95\textwidth]{pictures/ConfocalZeissPrinciple}
\end{center}
\caption{2-photons confocal microscope principle (Zeiss documentation)}
\label{fig:Confocal2photonsPrinciple}
\end{figure}

\begin{figure}[htb]
\begin{center}
\leavevmode
\includegraphics[width=0.95\textwidth]{pictures/PICmicroscope}
\end{center}
\caption{2-photons confocal microscope Zeiss 710 NLO in service in the Megason Lab}
\label{fig:MicMegason}
\end{figure}

\clearpage


%\input{ingenierie/Ingenierie}


% Chapitre sur le rapport de recherche :


\chapter{Research report} 

\section*{Introduction}

\subsection*{}
I am presenting in this chapter, the research project I have been working on during the PFE. I'have first undergone a bibliographic step, to acquire knowledge about the matter. Thatnks to this research, I could determine which problems are still to be solved. Then i tried to find appropriate solutions to the problems encountered.

The main difficulty in research, is that very often, there is no solution yet to the problem, in the domain. Therefore, I'have had to do a bibliography, in order to determine the state of the art, and eventually learn some new theories.

The image processing domain is particular: many algorithms are invented and published, but very few are available nor applicable to diverse images.
Thus, there is a re-programming and evaluation step, prior to any innovation.

\subsection*{}

While in the Megason Lab, I have been working on segmenting fluorescent microscopy images.
Datas are four-dimensional (space and time), and represent regions (ear, brain...) of a developing zebra fish.
It is possible to visualy distinc nuclei and membranes of cells. Those elements constitute the basis of the model that we try to create.
Therefore, we have to be able to detect and track every cell across time. The model will also have to integrate morphological informations of each cell.

The creation of this model is a thesis subject : cells lineage registration in microscopy, during which I would like to extend my work.

Prior to arriving in the laboratory, I have been working on cell membrane. I have then concentrated my researches on cell nuclei detection and localization.




%
%  STATE OF THE ART
%

\section{State of the art in the megason lab}


\subsection{Data analysis}

We are working on fluorescent microscopy images acquired with a 2-photons confocal microscope. The acquisition process produces huge datasets that we must process efficiently.
Here is a small description of the images :
\begin{figure}[htb]
\begin{center}
\begin{tabular}{|c|c|c|}
\hline Dimension & Size & Resolution \\ 
\hline x (space) & 1024 & 0.24 um \\ 
\hline y (space) & 1024 & 0.24 um \\ 
\hline z (space) & 70 & 1 um \\ 
\hline t (time) & 700 & 2 min\\ 
\hline \multicolumn{3}{|c|}{ 2 intensity channels} \\ 
\hline
\end{tabular} 
\end{center}
\caption{Megason Lab typical fluorescent microscopy dataset}
\label{tab:DataSizes}
\end{figure}
The table~\ref{tab:DataSizes}, shows that the datasets are huge ( the intensity values are coded with {\verb+unsigned char+}, so that a full two channel dataset is approximatively  10.5 Gbit).

Those datasets include nuclei and membrane information in two different intensity channels,and in the future, will include more channel with other biological markers. The imaging technique is point to point, and a mechanical displacement is involved for moving from on point to another (mirror displacement in the x-y plan, and stage displacement on the z axis). This leads to several artefacts and drawbacks :

\subsubsection{interlacing artefact}
This artefact is due to the microscope's raster scanning: it scans a line, following the x axis, and then increment on the y axis and scans another line.
The uncertainty on the x axis leads to interlacing of successive lines as illustrated figure~\ref{fig:InterlacingArtefact}.
This interlacing is also due to displacement uncertainties while scanning a line : as displacement varies, pixel width varies !
Thus this interlacing artefact is not isotropic: a line may seem to be translated of a positive factor on the right of the image, and of a negative factor on the left !

\TODO{figure illustrating interlacing problem theory}
\begin{figure}[htb]
  \centering
  \subfloat[nuclei channel]{\label{fig:InterlaceNuclei}\includegraphics[width=0.7\textwidth]{pictures/InterlacingNuclei}}\\
  \subfloat[membrane channel]{\label{fig:InterlaceMembrane}\includegraphics[width=0.7\textwidth]{pictures/InterlacingMembrane}}                
  \caption{Interlacing artifact on 2 photon confocal images. Views are x-y cuts of a 3D volume}
  \label{fig:InterlacingArtefact}
\end{figure}



\subsubsection{Anisotropy of images}


The datasets acquired in the Megason lab are anisotropic. As in the medical imaging domain, specialists are used to analyse images "slice by slice"
instead of considering them as a volume. They are worried about having a very detailed image in the x-y plan, and don't really worry about the other dimension.
This way of seeing things is wrong for images processing. A completely anisotropic image is very hard to process.
Non existing structures appear in the third dimension as illustrated figure~\ref{fig:Anysotropy}.
If a human brain with its understanding of the data may be able to interpret these structures, they lead most common algorithms to failure.

We can notice that nuclei seem to be stucked together and membrane's closure is almost always missing. This gives challenging a priori introduction problems.
\TODO{insert anisotropy examples}


\subsubsection{Noise}

There is much noise present in the datasets. Both membrane channel and nuclei channel are populated with non Gaussian noise.
The noise comes both from the microscope and electronics acquisition equipments, and from the biological sample that can be fluorescent in random locations.
The acquisition noise along z axis is decorrelated.
\TODO{get histograms of noise}

It could be interesting to study the response of the system to one small luminous point. The problem comes from the fact that depending on the phosphor used, the noise's statistics will differ.
\TODO{show noise in membrane and in nuclei}

\subsubsection{Low resolution of images}

The imaging process, combined with a too low resolution, leads to merged or missing structures. The nuclei channel displays very often clusters of merged nuclei when the membrane channel has information holes.
\TODO{show merged nuclei and holes in membrane}

\subsubsection{Point spread function}

The impulse response of the microscope is a noisy point spread function. That adds a deconvolution problem that can be treated together with the denoising problem. For simplicity reasons, searchers image with a very low resolution along the z axis, as the impulse function is spread out a lot along this axis.
\TODO{point spread function illustration}

\subsubsection{Complex structures}

The images we get are complex assembly of cells.
Some area being full of membrane, and others full with inter cellular liquid.
This is not a simple microscope slide with some well known cells, but a whole developing organism.


\TODO{show some fucked up functions}









\subsection{The megason lab imaging pipeline}

Dr Kishore Mosaliganti has been working for two years in the Megason lab, in order to develop new segmentation methods for fluorescent images.
He has been working on experiencing and developing diverse algorithms for nuclei and membrane detection and segmentation.
As presented is the previous section, the datasets in the Megason Lab are extremely challenging:
they are huge and present important drawbacks (resolution, noise) as they are provided for visual processing and not computer assisted processing.

As these datasets are very big, and four dimensional, it is not possible to use Matlab for easy prototyping and experiencing : 
most of the time, the very low resolution third dimension will considerably modify the results.
There is also a time issue : the datasets must be treated in an acceptable amount of time,
and that prevents us from using Matlab language that does not provide every optimized function that we need for 3D images visualization and processing.

This forces us to program and prototype in {\C++}. The standard library used for image processing in the Megason Lab is ITK. Kishore's algorithm are all coded with this library.
The data processing in ITK is represented by pipelines : a series of connected filters that perform image processing tasks.


Kishore studied several pipelines for nuclei and membrane segmentation.
For nuclei segmentation, an approach based on detection of nuclei, and region growing with the level set theory was used last year.
This year, a new approach based on nuclei detection and watershed algorithms is being used.
The algorithm used are standard and contour based for the detection of nuclei.
This leads to many errors in detection, due to the poor image quality. Right now, these errors are compensated after the segmentation step.

For membrane segmentation, Kishore is proposing a very good denoising technique based on anisotropic diffusion and tensor voting.
The reconstruction of the membrane structure is effective, even in low quality images. A segmentation step has to de implemented in top of it.

\subsection{The problem I am addressing}

I mainly focused on membrane segmentation prior to arriving in the Megason lab. This is a challenging problem, as shown by the data analysis : the membrane data is incomplete, noisy, three dimensional and anisotropic.

Having no prior knowledge on Kishore's data denoising, I focused on level sets techniques to segment the membrane.
The level set theory provides a very flexible framework for segmentation problems.
This is a generalisation of the fast marching algorithm. They are easily extensible to any dimension.
The drawbacks are that they are computationally intensive, and require theory learning prior to application.

Similarly to region growing algorithms, level set algorithms can be initialized to provide prior information.
I focused on growing a levelset function from cells center. For that purpose, we need a good cell localization.
I have been studying levels sets in Creatis, for the purpose of cell segmentation and had some results in two dimensions.
I was the considering that nuclei were already correctly detected and segmented, but when i arrived in the Megason Lab, I realized that before trying to segment the membrane using the techniques I studied in Creatis, I would need a correct cell nuclei detection.
We decided with Kishore that there was improvements to be brought to the cell nuclei detection algorithms.
















\fergfzgrertge 






ttttt

In the beginning of the internship, a new technique for nuclei detection was released.
As the one used in the Megason lab was not performing well enough, Kishore was willing to try and evaluate this new technique.



%
%  SEEDING
%


\section{Cell nuclei detection}

Cell nuclei shapes varies from spherical to curved ellipsoidal. As introduced above, the microscopy images are challenging, they are very noisy and anisotropic.
The main encountered difficulty is clustered nuclei that result from the point spread function of the microscope and the low resolution in the third dimension.

The algorithms used in the Megason lab for nuclei and membrane segmentation are based on an initialisation inside the nuclei.
Right now, a composite algorithm is used for detecting these nuclei. This algorithm is based on a combination of contour based algorithms :
Hough transform, \TOTO{Be clear : WHAT IS ZE TECHNIQUE? reading code is not instructive}
Those method provide us with landscapes (accumulation map for Hough transform, gradient vector flow tracking sinks, and finaly the intensity of smoothed nucleis) that we sum to combine the information fromo several sources. A local maxima detection is then performed on this resulting landscape.
This method gives good enough results for starting a watershed algorithm. Segmented regions that obviously don't correspond to a cell nucleus are eliminated. The final result of the segmentation process is evaluated.
\TODO{cite Kishore papers ?}


This is clear that for nuclei detection based segmentation methods, the nuclei detection has to be improved.
For that purpose we first implemented a new algorithm based on the Laplacian of Gaussian method.
Then we created an evaluation framework, to compare the results given by this new algorithm, and Kishore's.
We finally proposed a new method taking advantage of the cell membrane information.



\subsection{Bibliography on cell nuclei detection}

There are several papers dealing with cell nuclei detection, but very few are based on confocal images.
Another difficulty when doing such bibliography is the plethora of journals and the fact that Harvard Medical School doesn't have subscription for many imaging journals.
This restrains the scope of the bibliography.
I found several articles providing evaluation for their cell detection algorithm :
\begin{itemize}
  \item Constantinos G. et al\cite{•}, in his article about automatic counting of cancer cell nuclei in tissue sections propose an evaluaiton method based on two observers.
  They count the number of cells in ten images and the average number of cell counting error of the algorithm in percent the average number of cell present in the images is given (150). No information is provided on evaluating the cell nuclei position, or on the type of error : over detection or under detection ? The data are histological sections (2D).
  \item Gang Li et al\cite{}, in his article about 3D cell nuclei segmentation based on gradient flow tracking, presents a method for detecting and segmenting cell nuclei in 3D confocal
  datasets. The evaluation is done after the segmentation step and includes an elimination of detected nuclei that are too small. 
  They present the number of over segmented and under segmented cell nuclei, on a set of four images.
  \item P.S. Umesh Adiga et al\cite{}, in his article about segmentation of 3D histo-pathological images, present a method for segmenting cells in histological images 
  (which somehow close to nuclei in fluorescent images).
  The idea is to segment cells, using a watershed algorithm initialized on a simple cell center detection based on enhanced data (using morphological mathematics) and thresholding.
  The watershed algorithm is then run on such data and provides over segmented results. Those over segmented results are improved using constrained region merging. 
  The evaluation is done on 15 images and provides the original number of cells, and the number of detected cells for this algorithm.
  Two other algorithms are run on the same data to provide a comparison.
  \item Norberto Malpica et al\cite{•}, in his article about segmentation of clustered nuclei provides an evaluation based on two set of data : a training set (to adjust parameters of the algorithm), and a test set
\end{itemize}


Automatic detection of neurons in large cortical slices


%A "Plan of the internship"
%1 Goal (segment cells & membrane)
%2 Data presentation
%3 Where are we ? (membrane seg~0 cell seg needs lots of improvements)
%priority seeding imposed
%
%B Cell nuclei detection
% Intro
%bad seeding as of now, new ago, need for evaluation
%
%2 Biblio
%biblio on seeding evaluation
%
%3 Implementation
%implemantation … hard very hard (go through other's code forgotten already...)
%
%4 proposing
%new ago proposition new approach (based on membrane information)
%
%5 testing
%result of evaluation
%
%6 difficulties and future
%implementation big datasets 3D …
%future : merge results ? first : bad results everywhere : no more info for merging seedings...
%other ago based on wavelets? (as an opening)
%
%C Membrane segmentation
%  intro
%challenging need good cell detection...(but none present yet)
%1 biblio
%2 implementation (only 2D)
%3 result (only 2D)
%long very long but implementation can be greatly improved
%4 future
%not based on cell detection (start growing on the membrane)



%
%
%
%
%
%\section{Segmentation de la membrane cellulaire par ensembles de niveaux}
%
%Le but initial du PFE etait la segmentation de la membrane cellulaire. il s'agit d'une fine membrane séparant les multiples cellules. Elle s'étend sur tout le spécimen à analyser. Il s'agit donc d'un volume important et complexe.
%
%\subsection{Étude du problème}
%
%J'ai tout d'abord cherché à comprendre le problème posé : sur quelles données allaient se baser la détection, existe-t'il des solutions pour segmenter ce genre de données.
%
%\subsubsection{Les données}
%Les images sont acquises a travers un système optique. L'excitation par un laser entraine la fluorescence de certaines parties de la cellule, marquées par une molécule émettant de la lumière dans un spectre dépendant du marqueur utilisé.
%
%Le système a donc une réponse impulsionelle bien visible dans les données. Un point correspond grossièrement a une gaussienne étalée dans les trois dimensions de l'espace, et plus particulièrement selon l'axe perpendiculaire au plan de focalisation.
%
%Il existe aussi un bruit dû au dispositif électronique d'acquisition. De plus, la fluorescence n'étant pas répartie de manière homogène, il existe des "trous" et de la saturation dans les données.
%\TODO{inserer des images illustrant les problemes}
%
%
%J'ai choisi de me focaliser sur trois difficultés afin de trouver des solutions :
%\begin{description}
%  \item [problème du bruit] : quel filtrage appliquer aux images, afin de les débruiter.
%  \item [problème de l'absence de données] : comment introduire des à priori de forme de la membrane
%  pour palier à l'absence d'information ?
%  \item [problème de la non homogénéité des intensités]  : comment segmenter un objet
%  qui n'occupe pas les mêmes intensités selon sa position dans l'espace.
%\end{description}
%
%
%\subsection{Débruitage des données}
%Le bruit présent sur les images n'est pas gaussien. Il n'est pas répartis de la même manière dans toute l'image non plus. Je me suis donc focalisé sur des techniques de débruitage telles que le filtre médian, et plus généralement des filtres morphologiques.
%
%Le filtre médian donne de bons résultats, pour un temps de calcul inférieur aux filtres morphologiques (reconstruction par dilatation/erosion)
%
%\subsection{Segmentation de la membrane}
%
%\subsubsection{Utilisation de la théorie des ensembles de niveaux}
%
%L'outil choisi pour segmenter la paroi cellulaire, est base sur les ensembles de niveaux (levels-sets). Cette théorie consiste en l'évolution d'un front. Cette évolution est représentée par une fonction implicite qui évolue itérativement. Le front (bords de la zone segmentée) est souvent représenté par le niveau zéro de cette fonction implicite.
%Les Level sets, au travers de leur critère d'évolution, permettent d'avoir une grande flexibilité quand aux mesures a considerer lors de l'evolution du front. Cette évolution est représentée par un critère d'énergie, le problème de segmentation par level set est donc un problème d'optimisation.
%
%\subsection{des idees}
%
%
%
%
%
%
%idee de la mediane
%idee morphologie
%idee localisation
%\subsection{resultats}
%idee mediane
%idee morphologie
%idee localisation
%\subsection{travail futur}
%rapidite
%
%
%
%\section{Detection et localisation des cellules}
%
%Nous basons nos méthodes de segmentation sur une initialisation au centre des cellules. Nous avons donc besoin de détecter un maximum de cellules afin de trouver un point a l'intérieur de ces dernières. Des methodes ont ete proposees, cependant, chacune est adaptee a un type d'image particulier.
%Cels algorithmes de détection sont aussi souvent appelles algorithmes de "seeding" car ils permettent d'obtenir des points a partir desquels une segmentation peut etre initialisee, afin de delimiter les bordures des noyaux, ou les membranes cellulaires.
%
%\subsection{demarche}
%
%Nous avons developpe une methode combinant l'information provenant des noyaux et de la membrane des cellules. Cette methode doit etre evaluee, donc comparee a d'autres methodes existantes. Ce processus d'evaluation nous permettra aussi de trouver les points forts et les points faibles des algorithmes. Nous pourrons ainsi eventuellement utiliser des techniques de fusion d'information pour combiner les resultats de differents algorithmes.
%La creation d'un "framework" d'evaluation passe donc par plusieurs etapes : l'implementation des algorithmes existants, afin de les tester sur des images synthetiques puis reelles, la creation de criteres d'evaluation appropories, et l'observation des resultats. Nous avons aussi initié un travail afin de proposer une nouvelle methode de detection de cellules basee sur la decomposition en ondelettes.
%
%
%\subsection {description des algorithmes evalues}
%
%
%
%\subsubsection{chaine de traitement de l'image}
%
%partie commune
%Nous nous focalisons sur une classe d'algorithmes traitant l'information issue de l'image des noyaux cellulaires, apres une detection des zones d'interet (binarisation de l'image). Ces algorithmes fonctionnent aussi souvent avec une extraction de maxima locaux en dernier traitement.
%Nous choisissons d'utiliser la même binarisation, et la meme methode d'extraction de maximas locaux pour les deux algorithmes afin de focaliser l'etude sur la technique de detection des centres des noyaux.
%
%\subsubsection{description des algorithmes}
%\paragraph{le Laplacien de la Gaussienne ameliore}
%Nous avons decide d'implementer l'algorithme presente dans \cite{al2009improved}. La methode utilisee est celle du Laplacien de la Gaussienne (LoG). Une methode eprouveee qui s'est montree tres robuste dans d'autres applications telles la detection de points de reperes pour le recallage photographique.
%
%
%
%\paragraph{Kishore}
%
%\TODO{Ask Kishore more infos}
%
%
%\subsection {evaluation}
%%\begin{tabular}{|c|c|c|c|c|}
%%\hline  & Matching & UnMatching & Missed & Accuracy \\ 
%%\hline A1 & 10 & 3 & 1 & 71% \\ 
%%\hline A2 & 9 & 2 & 3 & 64% \\ 
%%\hline 
%%\end{tabular} 
%
%
%
%\subsection {conclusion}
%
%
%\subsection {proposition}
%
%
%
%\subsection {planning}
%
%
%\subsection{resultats}
%
%\subsection{proposition}
%
%
%
%


%
%
%
%\chapter{Conclusion du PFE}
%
%
%\section{Conclusion de la partie ingénierie}
%
%Durant ce PFE, j'ai travaillé sur des projets très variés.
%Ces projets étaient motivés par des objectifs scientifiques et nécessitaient chacun un haut niveau de technicité.
%
%J'ai eu l'occasion d'apprendre énormément,
%notamment dans le domaine des sciences informatiques,
%et du traitement de l'image.
%
%J'ai aussi découvert le milieu de la recherche, dont l'organisation est
%totalement différente de celle du milieu industriel.
%La hiérarchie, et les intérêts de chaque scientifique sont moins apparents que dans une entreprise et ce PFE a été riche en enseignements sur ce plan.
%Je suis parvenu à très bien m'intégrer dans l'équipe du Megason Lab en général,
%et à établir de bonnes relations de travail avec les divers scientifiques
%impliqués dans les projets sur lesquels j'ai travaillé.
%
%Cela m'a permis d'acquérir une base solide pour mener des recherches dans ce domaine.
%Après une importante phase d'apprentissage, je maitrise maintenant des librairies de renom
%(ITK, VTK,Qt), les systèmes unix, et la gestion de versions.
%
%J'ai pu aussi établir des contacts dans la communauté
%des traiteurs d'image par le biais de collaborations
%(Matthiew MacCormick), ou de d'"ateliers" (workshops) internationaux
%(participation à la NAMIC
%\footnote{La National Alliance for Medical Image Computing (NAMIC), est un réseau de professionnels de traitement d'images médicales. Ce réseau organise des conférences et des ateliers de travail ("workshop"), durant lesquels les scientifiques peuvent collaborer.}
%summer project week, au Massachusetts Institute of Technologies).
%
%
%\section{Projet professionnel}
%
%Ce PFE s'inscrit dans un projet professionnel construit durant ma scolarité à l'INSA de Lyon.
%Mon inscription à l'INSA de Lyon a été grandement motivée par l'ouverture de l'école à l'international. J'ai ainsi effectue mon stage ouvrier en Afrique du Sud. Profitant d'une première expérience professionnelle à l'international.
%
%Lors de ma seconde année a l'INSA de Lyon, j'ai choisi l'option SCiences et ANglais (SCAN).
%Cette filière regroupe les élèves ayant un niveau suffisant pour pouvoir suivre la formation généraliste en anglais.
%Elle s'accompagne d'une bourse pour un bref séjour linguistique dans une université étrangère.
%J'ai profité de ce financement pour aller au Trinity College en Irlande où
%j'ai pu avoir ma première expérience académique à l'étranger.
%
%J'ai ensuite choisi le département
%Génie Électrique qui permet de bénéficier d'un grand panel de compétences,
%notamment dans des domaines rattachés à l'électronique et l'informatique.
%
%
%Continuant l'expérience internationale, j'ai effectué mon stage industriel au
%Fraunhofer Institute, Center for Manufacturing Innovation, aux États Unis.
%J'ai été impliqué dans un projet de grande ampleur, financé par le département de la défense américain.
%Ce stage m'a énormément apporté en plus de l'expérience technique : il s'agissait d'une immersion dans la culture américaine, et j'ai eu une première expérience managériale en dirigeant une équipe d'ouvriers.
%C'est après ce stage que j'ai passé le test TOEIC validant un très bon niveau d'expression et de compréhension en anglais,
%avec un score de neuf cent quatre-vingt sur neuf cent quatre-vingt dix.
%
%
%Le PFE au Megason Lab possède des attraits indéniables :
%il s'agit d'un stage de recherche perfectionnant mes connaissances en traitement de l'image.
%Il s'agit aussi d'une expérience internationale dans une université renommée : Harvard Medical School.
%Ainsi, en plus de prendre plaisir à travailler en recherche, j'ouvre la porte à de nombreuses opportunités professionnelles.
%
%Comme je suis intéressé par les domaines techniques,
%je compte profiter de l'opportunité de poursuivre mes études pour obtenir un doctorat en mathématiques appliquées,
%au cours d'une thèse élaborée en collaboration avec le laboratoire CREATIS (INSA Lyon)
%et le laboratoire Megason (Harvard Medical School).
%Je passerai ainsi la moitié de mon temps en France, et l'autre moitié aux États Unis.
%En plus de pouvoir travailler sur un domaine passionnant,
%je pourrai ainsi bénéficier d'une équivalence avec un PhD (diplôme très reconnu à l'international). 
%
%Je compte ensuite mettre en valeur ma capacité a travailler dans un contexte international,
%dans des domaine à haute technicité pour trouver un emploi  dans le secteur privé.
%Je pourrai mettre en avant mes capacités en traitement du signal pour travailler dans le médical ou l'aéronautique.
%
%Mon ambition à long terme est de migrer vers des responsabilités managériales après avoir une très bonne maitrise de la technique.

\chapter*{Thanks}
\phantomsection
\addcontentsline{toc}{chapter}{Thanks}

I would like to thank, for their help and participation to my thesis, the following persons.

\begin{center}
Sean MEGASON\\
Kishore MOSALIGANTI\\
Arnaud GELAS\\
Olivier BERNARD\\
Rémy PROST\\
Matthiew MACCORMICK\\
Isabelle BLOCH\\
Nicolas RANNOU\\
Lydie SOUHAIT\\
Amelia GREEN\\
Nikolaus OBHOLZER\\
Ramil NOCHE\\
Raghav K. PADMANABHAN\\
Luis IBANEZ\\
\end{center}

And also the teams in Creatis LRMN, and in the Megason lab, for the great working ambiance I could find in these laboratories.

\appendix

%\input{ingenierie/AnnexesIngenierie}

\bibliographystyle{plain}
\bibliography{AntoBib}

\end{document}
